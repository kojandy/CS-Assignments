\documentclass{article}

\usepackage{fancyhdr}
\usepackage{extramarks}
\usepackage{amsmath}
\usepackage{amsthm}
\usepackage{amsfonts}
\usepackage{amssymb}
\usepackage{tikz}
\usepackage[plain]{algorithm}
\usepackage{algpseudocode}

\usetikzlibrary{automata,positioning}

%
% Basic Document Settings
%

\topmargin=-0.45in
\evensidemargin=0in
\oddsidemargin=0in
\textwidth=6.5in
\textheight=9.0in
\headsep=0.25in

\linespread{1.1}

\pagestyle{fancy}
\lhead{\hmwkAuthorName}
\chead{\hmwkClass\ (\hmwkClassInstructor\ | \hmwkClassTime): \hmwkTitle}
\rhead{\firstxmark}
\lfoot{\lastxmark}
\cfoot{\thepage}

\renewcommand\headrulewidth{0.4pt}
\renewcommand\footrulewidth{0.4pt}

\setlength\parindent{0pt}

%
% Create Problem Sections
%

\newcommand{\enterProblemHeader}[1]{
    \nobreak\extramarks{}{Problem \arabic{#1} continued on next page\ldots}\nobreak{}
    \nobreak\extramarks{Problem \arabic{#1} (continued)}{Problem \arabic{#1} continued on next page\ldots}\nobreak{}
}

\newcommand{\exitProblemHeader}[1]{
    \nobreak\extramarks{Problem \arabic{#1} (continued)}{Problem \arabic{#1} continued on next page\ldots}\nobreak{}
    \stepcounter{#1}
    \nobreak\extramarks{Problem \arabic{#1}}{}\nobreak{}
}

\setcounter{secnumdepth}{0}
\newcounter{partCounter}
\newcounter{homeworkProblemCounter}
\setcounter{homeworkProblemCounter}{1}
\nobreak\extramarks{Problem \arabic{homeworkProblemCounter}}{}\nobreak{}

%
% Homework Problem Environment
%
% This environment takes an optional argument. When given, it will adjust the
% problem counter. This is useful for when the problems given for your
% assignment aren't sequential. See the last 3 problems of this template for an
% example.
%
\newenvironment{homeworkProblem}[1][-1]{
    \ifnum#1>0
        \setcounter{homeworkProblemCounter}{#1}
    \fi
    \section{Problem \arabic{homeworkProblemCounter}}
    \setcounter{partCounter}{1}
    \enterProblemHeader{homeworkProblemCounter}
}{
    \exitProblemHeader{homeworkProblemCounter}
}

%
% Homework Details
%   - Title
%   - Due date
%   - Class
%   - Section/Time
%   - Instructor
%   - Author
%

\newcommand{\hmwkTitle}{Homework\ \#9}
\newcommand{\hmwkDueDate}{December 15, 2016}
\newcommand{\hmwkClass}{CS204}
\newcommand{\hmwkClassTime}{Section A}
\newcommand{\hmwkClassInstructor}{Prof. Sungwon Kang}
\newcommand{\hmwkAuthorName}{Ohjun Kwon}

%
% Title Page
%

\title{
    \vspace{2in}
    \textmd{\textbf{\hmwkClass:\ \hmwkTitle}}\\
    \normalsize\vspace{0.1in}\small{Due\ on\ \hmwkDueDate\ at 11:59pm}\\
    \vspace{0.1in}\large{\textit{\hmwkClassInstructor\ | \hmwkClassTime}}
    \vspace{3in}
}

\author{\textbf{20160051 \hmwkAuthorName}}
\date{}

\renewcommand{\part}{\textbf{\large (\alph{partCounter})}\stepcounter{partCounter}\\}

%
% Various Helper Commands
%

% Useful for algorithms
\newcommand{\alg}[1]{\textsc{\bfseries \footnotesize #1}}

% For derivatives
\newcommand{\deriv}[1]{\frac{\mathrm{d}}{\mathrm{d}x} (#1)}

% For partial derivatives
\newcommand{\pderiv}[2]{\frac{\partial}{\partial #1} (#2)}

% Integral dx
\newcommand{\dx}{\mathrm{d}x}

% Alias for the Solution section header
\newcommand{\solution}{\textbf{\large Solution}}

% Probability commands: Expectation, Variance, Covariance, Bias
\newcommand{\E}{\mathrm{E}}
\newcommand{\Var}{\mathrm{Var}}
\newcommand{\Cov}{\mathrm{Cov}}
\newcommand{\Bias}{\mathrm{Bias}}

\newcommand\numberthis{\addtocounter{equation}{1}\tag{\theequation}}

\begin{document}

\maketitle

\pagebreak
\begin{homeworkProblem}
    The overall participants that are going to win the prize will be 3 people out of 100 participants.
    Therefore, the probability that a participant of the contest wins one of prizes will be $3/100$.
    $\therefore 3/100$
\end{homeworkProblem}

\begin{homeworkProblem}
    % prob 2
\end{homeworkProblem}

\begin{homeworkProblem}
    \part
    First, we can think the total permutation of a, a, 2, 3, which is $4!/2!$.
    Then, we can put 1 and 4 in the position of a's in order of 1, 4,
    because the statement says ``1 precedes 4''.
    For example, 3a2a makes the sequence 3124.
    $\therefore 12$
    \\

    \part
    We can think this problem in same way as the \textbf{Problem 3 (a)}.
    If we put 4 and 1 consecutively in the a's positions, we can get sequences that satisfies the given statement.
    $\therefore 12$
    \\

    \part
    We can think this problem as picking 3's position in \_4\_a\_a\_, and put 1 and 2 in a's position randomly.
    Therefore, there is three possible 3's seats, and two ways to put 1 and 2.
    $\therefore 3\times 2=6$
    \\

    \part
    We can think this problem as putting 4 in front of the random permutations of $\{1,2,3\}$.
    $\therefore 3!=6$
    \\

    \part
    We can think this problem as putting 4 and 3 consecutively in a's positions, and 2 and 1 consecutively in b's positions
    in the permutation of aabb.
    For example, if we get abba out of the permutation of aabb, it means 4213.
    $\displaystyle \therefore \frac{4!}{2!2!}=6$
\end{homeworkProblem}

\begin{homeworkProblem}
    % prob 4
\end{homeworkProblem}

\begin{homeworkProblem}
    \begin{align*}
        E[\mbox{score}]&=50\times 2\times 0.9+25\times 4\times 0.8\\
        &=90+80\\
        &=170
    \end{align*}
\end{homeworkProblem}

\begin{homeworkProblem}
    % prob 6
\end{homeworkProblem}

\begin{homeworkProblem}
    % prob 7
\end{homeworkProblem}

\begin{homeworkProblem}
    % prob 8
\end{homeworkProblem}

\begin{homeworkProblem}
    % prob 9
\end{homeworkProblem}

\begin{homeworkProblem}
    % prob 10
\end{homeworkProblem}
\end{document}
