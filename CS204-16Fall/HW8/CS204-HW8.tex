\documentclass{article}

\usepackage{fancyhdr}
\usepackage{extramarks}
\usepackage{amsmath}
\usepackage{amsthm}
\usepackage{amsfonts}
\usepackage{amssymb}
\usepackage{tikz}
\usepackage[plain]{algorithm}
\usepackage{algpseudocode}

\usetikzlibrary{automata,positioning}

%
% Basic Document Settings
%

\topmargin=-0.45in
\evensidemargin=0in
\oddsidemargin=0in
\textwidth=6.5in
\textheight=9.0in
\headsep=0.25in

\linespread{1.1}

\pagestyle{fancy}
\lhead{\hmwkAuthorName}
\chead{\hmwkClass\ (\hmwkClassInstructor\ | \hmwkClassTime): \hmwkTitle}
\rhead{\firstxmark}
\lfoot{\lastxmark}
\cfoot{\thepage}

\renewcommand\headrulewidth{0.4pt}
\renewcommand\footrulewidth{0.4pt}

\setlength\parindent{0pt}

%
% Create Problem Sections
%

\newcommand{\enterProblemHeader}[1]{
    \nobreak\extramarks{}{Problem \arabic{#1} continued on next page\ldots}\nobreak{}
    \nobreak\extramarks{Problem \arabic{#1} (continued)}{Problem \arabic{#1} continued on next page\ldots}\nobreak{}
}

\newcommand{\exitProblemHeader}[1]{
    \nobreak\extramarks{Problem \arabic{#1} (continued)}{Problem \arabic{#1} continued on next page\ldots}\nobreak{}
    \stepcounter{#1}
    \nobreak\extramarks{Problem \arabic{#1}}{}\nobreak{}
}

\setcounter{secnumdepth}{0}
\newcounter{partCounter}
\newcounter{homeworkProblemCounter}
\setcounter{homeworkProblemCounter}{1}
\nobreak\extramarks{Problem \arabic{homeworkProblemCounter}}{}\nobreak{}

%
% Homework Problem Environment
%
% This environment takes an optional argument. When given, it will adjust the
% problem counter. This is useful for when the problems given for your
% assignment aren't sequential. See the last 3 problems of this template for an
% example.
%
\newenvironment{homeworkProblem}[1][-1]{
    \ifnum#1>0
        \setcounter{homeworkProblemCounter}{#1}
    \fi
    \section{Problem \arabic{homeworkProblemCounter}}
    \setcounter{partCounter}{1}
    \enterProblemHeader{homeworkProblemCounter}
}{
    \exitProblemHeader{homeworkProblemCounter}
}

%
% Homework Details
%   - Title
%   - Due date
%   - Class
%   - Section/Time
%   - Instructor
%   - Author
%

\newcommand{\hmwkTitle}{Homework\ \#8}
\newcommand{\hmwkDueDate}{December 8, 2016}
\newcommand{\hmwkClass}{CS204}
\newcommand{\hmwkClassTime}{Section A}
\newcommand{\hmwkClassInstructor}{Prof. Sungwon Kang}
\newcommand{\hmwkAuthorName}{Ohjun Kwon}

%
% Title Page
%

\title{
    \vspace{2in}
    \textmd{\textbf{\hmwkClass:\ \hmwkTitle}}\\
    \normalsize\vspace{0.1in}\small{Due\ on\ \hmwkDueDate\ at 11:59pm}\\
    \vspace{0.1in}\large{\textit{\hmwkClassInstructor\ | \hmwkClassTime}}
    \vspace{3in}
}

\author{\textbf{20160051 \hmwkAuthorName}}
\date{}

\renewcommand{\part}{\textbf{\large (\alph{partCounter})}\stepcounter{partCounter}\\}

%
% Various Helper Commands
%

% Useful for algorithms
\newcommand{\alg}[1]{\textsc{\bfseries \footnotesize #1}}

% For derivatives
\newcommand{\deriv}[1]{\frac{\mathrm{d}}{\mathrm{d}x} (#1)}

% For partial derivatives
\newcommand{\pderiv}[2]{\frac{\partial}{\partial #1} (#2)}

% Integral dx
\newcommand{\dx}{\mathrm{d}x}

% Alias for the Solution section header
\newcommand{\solution}{\textbf{\large Solution}}

% Probability commands: Expectation, Variance, Covariance, Bias
\newcommand{\E}{\mathrm{E}}
\newcommand{\Var}{\mathrm{Var}}
\newcommand{\Cov}{\mathrm{Cov}}
\newcommand{\Bias}{\mathrm{Bias}}

\newcommand\numberthis{\addtocounter{equation}{1}\tag{\theequation}}

\begin{document}

\maketitle

\pagebreak
\begin{homeworkProblem}
    We can make $m\times n$ relations out of two sets $A$ and $B$, becuase we have to pick one element in each sets.
    There are m elements in the set $A$ and n elements in the set $B$,
    so by the multiplication rule there are $m\times n$ binary relations.

    In other words, a binary relation is an ordered pair(an element from the set $A\times B$),
    so by the multiplication rule, there are $|A|\cdot |B|$ ordered pairs,
    which is $m\times n$.
    
    $\therefore mn$
\end{homeworkProblem}

\begin{homeworkProblem}
    There are $2\cdot 2\cdot 2\cdot 2=2^4=16$ binary sequences if the sequence's length is 4,
    and we want to encode the decimal digits 0-9 with this sequences.
    We can calculate the distinct way of representing the digits with sequences by choosing a sequence
    for each digits respectively.
    We can choose a sequence for encoding the digit 0 within 16 binary sequences. After choosing that,
    we can choose another sequence for the digit 1 within the sequences excluding the previously selected sequence.
    Continuing this for the digits 0 through 9, we can calculate the number of distinct ways to encode the given decimal digits.

    $\therefore 16\cdot 15\cdot 14\dots 9\cdot 8\cdot 7={}_{16}P_{10}=29059430400$
\end{homeworkProblem}

\begin{homeworkProblem}
    \part
    In order to make a word have even number of 1s, we have to choose even numbers of position to put 1.
    Then, set selected position 1 and 0 for other positions. The numbers of the words will be
    ${}_nC_0+{}_nC_2+{}_nC_4+\dots$.
    We can calculate the expression with binomial expansion.
    From the Equation~\ref{eq:giv}, we can obtain Equation~\ref{eq:ob1} and Equation~\ref{eq:ob2} by
    substituting $x=1$ and $x=-1$ respectively.
    \begin{align}
        (1+x)^n&=\sum_{k=0}^n {n\choose k} x^k\label{eq:giv}\\
        2^n&=\sum_{k=0}^n {n\choose k}\label{eq:ob1}\\
        0&=\sum_{k=0}^n {n\choose k}(-1)^k\label{eq:ob2}
    \end{align}
    By adding Equation~\ref{eq:ob1} and Equation~\ref{eq:ob2}, we obtain Equation~\ref{eq:ob3} which is the answer we were looking for.
    \begin{align*}
        2^n&=2\left( {n\choose 0}+{n\choose 2}+{n\choose 4}+\dots \right) \\
        2^{n-1}&={n\choose 0}+{n\choose 2}+{n\choose 4}+\dots \numberthis \label{eq:ob3}
    \end{align*}
    $\therefore 2^{n-1}$
    \\

    \part
    Similar to \textbf{(a)}, we have to calculate ${}_nC_1+{}_nC_3+\dots$.
    We can calculate this by subtracting Equation~\ref{eq:ob2} from Equation~\ref{eq:ob1}.
    \begin{align*}
        2^n&=2\left( {n\choose 1}+{n\choose 3}+{n\choose 5}+\dots \right) \\
        2^{n-1}&={n\choose 1}+{n\choose 3}+{n\choose 5}+\dots
    \end{align*}
    $\therefore 2^{n-1}$
\end{homeworkProblem}

\begin{homeworkProblem}
    We can think this problem as choosing numbers in 1 to 5. For example, if we got H for the first trial.
    It can be considered equally with choosing 1 out of five numbers.
    Same way, if we get HHTTT as a result, we can interpret this result as we have chosen 1, 2 within 1 to 5.
    Using this of interpretation, we can answer the given questions.
    For the first two questions, we can think the problem as picking one and two numbers in 1 to 5.
    Therefore, the answers will be $_5C_1$ and $_5C_2$, respectively.
    $\therefore 5, 10$

    For the last problem, which is the generalization of the previous questions,
    we can think this problem as picking r numbers within n numbers.

    $\therefore {}_nC_r$
\end{homeworkProblem}

\begin{homeworkProblem}
    \begin{align*}
        {}_nC_{r-1}+{}_nC_r&=\frac{n!}{(r-1)!(n-r+1)!}+\frac{n!}{r!(n-r)!}\\
        &=\frac{n!\times r}{r!(n+1-r)!}+\frac{n!\times (n+1-r)}{r!(n+1-r)!}\\
        &=\frac{n!(r+n+1-r)}{r!(n+1-r)!}\\
        &=\frac{(n+1)!}{r!(n+1-r)!}\\
        &={}_{n+1}C_r
    \end{align*}
    \qedsymbol
\end{homeworkProblem}

\begin{homeworkProblem}
    There are 9 letters in the word \textbf{INANENESS}, so there are 9! ways to rearrange the letters.
    However, the word contains duplicate letters such as \textbf{N} and \textbf{E},
    so there will be duplicates in 9! ways of rearrangements.
    We can count out these duplicates by counting the possible rearrangements with the duplicate letters.
    For example, with the word \textbf{AAB}, there are only 3 possible rearrangements \textbf{AAB}, \textbf{ABA}, \textbf{BAA}.
    However, there are two duplicate letters \textbf{A}, and its number of possible rearrangements is 2!.
    Then, the overall number of possible rearrangements considering the duplicate letters will become $\frac{3!}{2!}=3$.
    Therefore, we can exclude duplicated rearrangements with dividing the possible rearrangements from the duplicate letters,
    and there are three \textbf{N}s, two \textbf{E}s, and two \textbf{S}s in the given word.

    $\displaystyle \therefore \frac{9!}{3!2!2!}=15120$
\end{homeworkProblem}

\begin{homeworkProblem}
    If we don't think about the symmetry, there will be $8!=40320$ ways of putting glasses.
    However, there will be duplicates if we count this way since we have to consider the symmetry.
    We can remove these duplicate counts by dividing the number with the possible duplicate of a one pattern.
    From a pattern like Table~\ref{tab:givpat}, we can make 3 other patterns out of the given pattern.
    As we can see in Table~\ref{tab:sapa}, we can make total 4 patterns including the original pattern from one pattern.
    The represented patterns in Table~\ref{tab:sapa} are rotated 180$^\circ$, flipped vertically, and flipped horizontally from left to right.
    In other words, we have counted 1 pattern as 4 pattern.
    Therefore, we have to divide $40320$ with $4$.
    \begin{table}[!htb]
        \begin{tabular}{| c | c | c | c |}
            \hline
            1&2&3&4\\
            \hline
            5&6&7&8\\
            \hline
        \end{tabular}
        \centering
        \caption{given pattern}
        \label{tab:givpat}
    \end{table}
    \begin{table}[!htb]
        \begin{tabular}{| c | c | c | c | c | c | c | c | c | c | c | c | c | c |}
            \cline{1-4}\cline{6-9}\cline{11-14}
            8&7&6&5&&4&3&2&1&&5&6&7&8\\
            \cline{1-4}\cline{6-9}\cline{11-14}
            4&3&2&1&&8&7&6&5&&1&2&3&4\\
            \cline{1-4}\cline{6-9}\cline{11-14}
        \end{tabular}
        \centering
        \caption{patterns that are considered same}
        \label{tab:sapa}
    \end{table}

    $\therefore 8!/4=10080$
\end{homeworkProblem}

\begin{homeworkProblem}
    All of the natural numbers can be divided into to even and odd.
    However, if we categorize three numbers,
    there will be a set which has at least two numbers in it by pigeonhole principle.
    If we add up these two numbers, the result will be even.
    ($\because \mbox{even}+\mbox{even}=\mbox{even}$, $\mbox{odd}+\mbox{odd}=\mbox{even}$)
    \qedsymbol
\end{homeworkProblem}

\begin{homeworkProblem}
    If there are $n$ number of classrooms and 12 different time slots are available,
    we can accommodate total $12n$ classes. The college wants to offer 250 differenct classes,
    so the number of possible classes should not be lesser than 250.
    $12n \geq 250 \Leftrightarrow n\geq 20.8333\dots$

    $\therefore$ There should be at least 21 number of classrooms.
\end{homeworkProblem}

\begin{homeworkProblem}
    There will be $100-12=88$ losing tickets, and these losing tickets will be divided in to 13 streaks with 12 winning tickets.
    Let $X$ be the losing tickets, $C$ be the group, and $f:X\to C$ be the function that returns the group $f(x)$
    which the owner of the losing ticket $x$ belongs to.
    Then, we have to find the maximum $k$ that satisfies $88>13(k-1)$.

    $\therefore k=7$
\end{homeworkProblem}
\end{document}
