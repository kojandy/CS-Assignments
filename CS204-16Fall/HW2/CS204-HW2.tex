\documentclass{article}

\usepackage{fancyhdr}
\usepackage{extramarks}
\usepackage{amsmath}
\usepackage{amsthm}
\usepackage{amsfonts}
\usepackage{tikz}
\usepackage[plain]{algorithm}
\usepackage{algpseudocode}
\usepackage{fitch}

\usetikzlibrary{automata,positioning}

%
% Basic Document Settings
%

\topmargin=-0.45in
\evensidemargin=0in
\oddsidemargin=0in
\textwidth=6.5in
\textheight=9.0in
\headsep=0.25in

\linespread{1.1}

\pagestyle{fancy}
\lhead{\hmwkAuthorName}
\chead{\hmwkClass\ (\hmwkClassInstructor\ | \hmwkClassTime): \hmwkTitle}
\rhead{\firstxmark}
\lfoot{\lastxmark}
\cfoot{\thepage}

\renewcommand\headrulewidth{0.4pt}
\renewcommand\footrulewidth{0.4pt}

\setlength\parindent{0pt}

%
% Create Problem Sections
%

\newcommand{\enterProblemHeader}[1]{
    \nobreak\extramarks{}{Problem \arabic{#1} continued on next page\ldots}\nobreak{}
    \nobreak\extramarks{Problem \arabic{#1} (continued)}{Problem \arabic{#1} continued on next page\ldots}\nobreak{}
}

\newcommand{\exitProblemHeader}[1]{
    \nobreak\extramarks{Problem \arabic{#1} (continued)}{Problem \arabic{#1} continued on next page\ldots}\nobreak{}
    \stepcounter{#1}
    \nobreak\extramarks{Problem \arabic{#1}}{}\nobreak{}
}

\setcounter{secnumdepth}{0}
\newcounter{partCounter}
\newcounter{homeworkProblemCounter}
\setcounter{homeworkProblemCounter}{1}
\nobreak\extramarks{Problem \arabic{homeworkProblemCounter}}{}\nobreak{}

%
% Homework Problem Environment
%
% This environment takes an optional argument. When given, it will adjust the
% problem counter. This is useful for when the problems given for your
% assignment aren't sequential. See the last 3 problems of this template for an
% example.
%
\newenvironment{homeworkProblem}[1][-1]{
    \ifnum#1>0
        \setcounter{homeworkProblemCounter}{#1}
    \fi
    \section{Problem \arabic{homeworkProblemCounter}}
    \setcounter{partCounter}{1}
    \enterProblemHeader{homeworkProblemCounter}
}{
    \exitProblemHeader{homeworkProblemCounter}
}

%
% Homework Details
%   - Title
%   - Due date
%   - Class
%   - Section/Time
%   - Instructor
%   - Author
%

\newcommand{\hmwkTitle}{Homework\ \#2}
\newcommand{\hmwkDueDate}{September 29, 2016}
\newcommand{\hmwkClass}{CS204}
\newcommand{\hmwkClassTime}{Section A}
\newcommand{\hmwkClassInstructor}{Prof. Sungwon Kang}
\newcommand{\hmwkAuthorName}{Ohjun Kwon}

%
% Title Page
%

\title{
    \vspace{2in}
    \textmd{\textbf{\hmwkClass:\ \hmwkTitle}}\\
    \normalsize\vspace{0.1in}\small{Due\ on\ \hmwkDueDate\ at 11:59pm}\\
    \vspace{0.1in}\large{\textit{\hmwkClassInstructor\ | \hmwkClassTime}}
    \vspace{3in}
}

\author{\textbf{20160051 \hmwkAuthorName}}
\date{}

\renewcommand{\part}[1]{\textbf{\large Part \Alph{partCounter}}\stepcounter{partCounter}\\}

%
% Various Helper Commands
%

% Useful for algorithms
\newcommand{\alg}[1]{\textsc{\bfseries \footnotesize #1}}

% For derivatives
\newcommand{\deriv}[1]{\frac{\mathrm{d}}{\mathrm{d}x} (#1)}

% For partial derivatives
\newcommand{\pderiv}[2]{\frac{\partial}{\partial #1} (#2)}

% Integral dx
\newcommand{\dx}{\mathrm{d}x}

% Alias for the Solution section header
\newcommand{\solution}{\textbf{\large Solution}}

% Probability commands: Expectation, Variance, Covariance, Bias
\newcommand{\E}{\mathrm{E}}
\newcommand{\Var}{\mathrm{Var}}
\newcommand{\Cov}{\mathrm{Cov}}
\newcommand{\Bias}{\mathrm{Bias}}

\begin{document}

\maketitle

\pagebreak

\begin{homeworkProblem}
\textbf{(a)}
All penguins are dangerous.

\textbf{(b)}
Some penguins are dangerous.

\textbf{(c)}
There is no penguin that is dangerous.

\textbf{(d)}
Some penguins are not dangerous.
\end{homeworkProblem}

\begin{homeworkProblem}
\textbf{(a)}
$(\forall x)(L(x)\to F(x))$

\textbf{(b)}
$(\forall x)(L(x)\land F(x))$
\end{homeworkProblem}

\begin{homeworkProblem}
\textbf{(a)}
$(\exists x)P(x)$

\textbf{(b)}
$(\forall x)(P(x)\to \lnot Q(x))$

\textbf{(c)}
$(\exists x)(\lnot P(x) \land \lnot Q(x))$
\end{homeworkProblem}

\begin{homeworkProblem}
\textbf{(a)}
$(\forall x)(\exists y)(N(x)\to P(x,y))$

\textbf{(b)}
\begin{align}
\lnot \forall x\exists y(N(x)\to P(x,y)) &\Leftrightarrow \exists x \lnot \exists y (N(x)\to P(x,y)) \\
&\Leftrightarrow \exists x \forall y \lnot (N(x)\to P(x,y)) \\
&\Leftrightarrow \exists x \forall y \lnot (\lnot N(x) \lor P(x,y)) \\
&\Leftrightarrow \exists x \forall y (\lnot \lnot N(x) \land \lnot P(x,y)) \\
&\Leftrightarrow \exists x \forall y (N(x) \land \lnot P(x,y))
\end{align}
(1): universal negation \\
(2): existential negation \\
(3): material implication \\
(4): De Morgan's law \\
(5): double negation \\

\textbf{(c)}
$\exists x \forall y (N(x) \land \lnot P(x,y))$

There is some integer x such that $x\neq 0$ and $xy\neq 1$ for all integers y.

\textbf{(d)}
\textbf{(b)} is true.

\textbf{Proof}

case \textbf{(a)}: for all integers x, y that satisfies $xy=1$ is uniquely exists as $y=1/x$, which does not belong to the integer set. The statement satisfies When x is $-1$ or $1$, but it does not cover all integers.

case \textbf{(b)}: if we select $2$ as $x$, then it satisfies two statement, $x\neq 0$ and $xy\neq 1$, for any integer y.
\end{homeworkProblem}

\begin{homeworkProblem}
\textbf{(a)}
For all traders who work at the Tokyo Stock Exchange, there exists a trader that makes less money than him/her.

\textbf{(b)}
There exists a trader $A$, that makes more money than any trader other than $A$.

\textbf{(c)}
\textbf{(a)} is an impossible statement, because there should be a minimum in a finite set(traders who work at the Tokyo Stock Exchange). Then, if we pick a minimum, then there will be no traders that satisfy the statement. Therefore, the minimum is the counter-example of the statement, and it proves that the \textbf{(a)} is a false statement. In the similar context, if we pick a trader who has a maximum income for $A$, there will be no trader who make more money other than $A$. It satisfies the statement \textbf{(b)}.
\end{homeworkProblem}

\begin{homeworkProblem}
Domain: $\mathbb{R}$

$R(x)$: x is rational.
\\

Given statement: $\forall x \forall y ((R(x)\land \lnot R(y))\to \lnot R(x+y))$

Negation: 
\begin{align}
\lnot \forall x \forall y ((R(x)\land \lnot R(y))\to \lnot R(x+y)) &\Leftrightarrow \exists x \lnot \forall y ((R(x)\land \lnot R(y))\to \lnot R(x+y)) \\
&\Leftrightarrow \exists x \exists y \lnot ((R(x)\land \lnot R(y))\to \lnot R(x+y)) \\
&\Leftrightarrow \exists x \exists y \lnot (\lnot(R(x)\land \lnot R(y))\lor \lnot R(x+y)) \\
&\Leftrightarrow \exists x \exists y (\lnot \lnot(R(x)\land \lnot R(y))\land \lnot \lnot R(x+y)) \\
&\Leftrightarrow \exists x \exists y ((R(x)\land \lnot R(y))\land R(x+y)) \\
&\Leftrightarrow \exists x \exists y (R(x)\land \lnot R(y)\land R(x+y))
\end{align}
(6): universal negation \\
(7): universal negation \\
(8): material implication \\
(9): De Morgan's law \\
(10): double negation \\
(11): associativity
\end{homeworkProblem}

\begin{homeworkProblem}
\textbf{(a)}
\begin{align}
\lnot (\exists x) (R(x) \land B(x)) &\Leftrightarrow (\forall x)\lnot (R(x)\land B(x)) \\
&\Leftrightarrow (\forall x)(\lnot R(x)\lor \lnot B(x)) \\
&\Leftrightarrow (\forall x)(R(x)\to \lnot B(x))
\end{align}
(12): existential negation \\
(13): De Morgan's law \\
(14): material implication \\

\textbf{(b)}
There is no triangle that is a right triangle and also has an obtuse angle.

\textbf{(c)}
For all triangles $x$, if $x$ is a right triangle, then $x$ does not have an obtuse angle.
\end{homeworkProblem}

\begin{homeworkProblem}
\textbf{(a)}
I will give a example by using predicates from \textbf{Problem 7}.

Domain: triangles \\
Predicates: \\
$R(x)$= $x$ is a right triangle.\\
$B(x)$= $x$ has an obtuse angle.\\

$(\exists x)(P(x)\land Q(x))$: There exists a triangle that is a right triangle and also has an obtuse angle.

$(\exists x)P(x)\land (\exists x)Q(x)$: There exists a triangle that is a right triangle, and there also exists a triangle that has an obtuse angle.\\

The first statement is false, because if a triangle has both right and obtuse angle, the sum of the interior angles exceeds $180^\circ$. However, as we know, the sum of the interior angles of a triangle should be $180^\circ$. For the second sentence, of course, there exists a right triangle, and a triangle that has an obtuse angle. Therefore, the second statement is true. The result of the statements are different, so they are not logically equivalent.\\

\textbf{(b)}
The given two statements can be translated into ordinary English as below.

$(\exists x)(P(x)\lor Q(x))$: There exists a triangle that is a right triangle or has an obtuse angle.\\
$(\exists x)P(x)\lor (\exists x)Q(x)$: There exists a triangle that is a right triangle, or there exists a triangle that has an obtuse angle.\\

The first and second sentences can be satisfied either a right triangle or a triangle that has an obtuse angle.
\end{homeworkProblem}

\begin{homeworkProblem}
\textbf{(a)}
$\exists x T(x),\forall x(T(x)\to P(x))\vdash \exists y(T(y)\land P(y))$
\[
\begin{nd}
\hypo{1}{\exists x T(x)}
\hypo{2}{\forall x(T(x)\to P(x))}
\open[a]
\have{3}{T(a)\to P(a)}\Ae{2}
\open
\hypo{4}{T(a)}
\have{5}{P(a)}\ie{3,4}
\have{6}{T(a)\land P(a)}\ai{4,5}
\close
\have{7}{T(a)\land P(a)}\Ee{1,4-6}
\close
\have{8}{\exists y(T(y)\land P(y))}\Ei{7}
\end{nd}
\]

\textbf{(b)}
$\forall x(P(x)\land Q(x))\vdash \forall xP(x)\land \forall yQ(y)$
\[
\begin{nd}
\hypo{1}{\forall x(P(x)\land Q(x))}
\open[t]
\have{2}{P(t)\land Q(t)}\Ae{1}
\have{3}{P(t)}\ae{2}
\have{4}{Q(t)}\ae{2}
\close
\have{5}{\forall x P(x)}\Ai{3}
\have{6}{\forall y Q(y)}\Ai{4}
\have{7}{\forall xP(x)\land \forall yQ(y)}\ai{5,6}
\end{nd}
\]
\end{homeworkProblem}
\end{document}