\documentclass{article}

\usepackage{fancyhdr}
\usepackage{extramarks}
\usepackage{amsmath}
\usepackage{amsthm}
\usepackage{amsfonts}
\usepackage{tikz}
\usepackage[plain]{algorithm}
\usepackage{algpseudocode}

\usetikzlibrary{automata,positioning}

%
% Basic Document Settings
%

\topmargin=-0.45in
\evensidemargin=0in
\oddsidemargin=0in
\textwidth=6.5in
\textheight=9.0in
\headsep=0.25in

\linespread{1.1}

\pagestyle{fancy}
\lhead{\hmwkAuthorName}
\chead{\hmwkClass\ (\hmwkClassInstructor\ | \hmwkClassTime): \hmwkTitle}
\rhead{\firstxmark}
\lfoot{\lastxmark}
\cfoot{\thepage}

\renewcommand\headrulewidth{0.4pt}
\renewcommand\footrulewidth{0.4pt}

\setlength\parindent{0pt}

%
% Create Problem Sections
%

\newcommand{\enterProblemHeader}[1]{
    \nobreak\extramarks{}{Problem \arabic{#1} continued on next page\ldots}\nobreak{}
    \nobreak\extramarks{Problem \arabic{#1} (continued)}{Problem \arabic{#1} continued on next page\ldots}\nobreak{}
}

\newcommand{\exitProblemHeader}[1]{
    \nobreak\extramarks{Problem \arabic{#1} (continued)}{Problem \arabic{#1} continued on next page\ldots}\nobreak{}
    \stepcounter{#1}
    \nobreak\extramarks{Problem \arabic{#1}}{}\nobreak{}
}

\setcounter{secnumdepth}{0}
\newcounter{partCounter}
\newcounter{homeworkProblemCounter}
\setcounter{homeworkProblemCounter}{1}
\nobreak\extramarks{Problem \arabic{homeworkProblemCounter}}{}\nobreak{}

%
% Homework Problem Environment
%
% This environment takes an optional argument. When given, it will adjust the
% problem counter. This is useful for when the problems given for your
% assignment aren't sequential. See the last 3 problems of this template for an
% example.
%
\newenvironment{homeworkProblem}[1][-1]{
    \ifnum#1>0
        \setcounter{homeworkProblemCounter}{#1}
    \fi
    \section{Problem \arabic{homeworkProblemCounter}}
    \setcounter{partCounter}{1}
    \enterProblemHeader{homeworkProblemCounter}
}{
    \exitProblemHeader{homeworkProblemCounter}
}

%
% Homework Details
%   - Title
%   - Due date
%   - Class
%   - Section/Time
%   - Instructor
%   - Author
%

\newcommand{\hmwkTitle}{Homework\ \#1}
\newcommand{\hmwkDueDate}{September 20, 2016}
\newcommand{\hmwkClass}{CS204}
\newcommand{\hmwkClassTime}{Section A}
\newcommand{\hmwkClassInstructor}{Prof. Sungwon Kang}
\newcommand{\hmwkAuthorName}{Ohjun Kwon}

%
% Title Page
%

\title{
    \vspace{2in}
    \textmd{\textbf{\hmwkClass:\ \hmwkTitle}}\\
    \normalsize\vspace{0.1in}\small{Due\ on\ \hmwkDueDate\ at 11:59pm}\\
    \vspace{0.1in}\large{\textit{\hmwkClassInstructor\ | \hmwkClassTime}}
    \vspace{3in}
}

\author{\textbf{20160051 \hmwkAuthorName}}
\date{}

\renewcommand{\part}[1]{\textbf{\large Part \Alph{partCounter}}\stepcounter{partCounter}\\}

%
% Various Helper Commands
%

% Useful for algorithms
\newcommand{\alg}[1]{\textsc{\bfseries \footnotesize #1}}

% For derivatives
\newcommand{\deriv}[1]{\frac{\mathrm{d}}{\mathrm{d}x} (#1)}

% For partial derivatives
\newcommand{\pderiv}[2]{\frac{\partial}{\partial #1} (#2)}

% Integral dx
\newcommand{\dx}{\mathrm{d}x}

% Alias for the Solution section header
\newcommand{\solution}{\textbf{\large Solution}}

% Probability commands: Expectation, Variance, Covariance, Bias
\newcommand{\E}{\mathrm{E}}
\newcommand{\Var}{\mathrm{Var}}
\newcommand{\Cov}{\mathrm{Cov}}
\newcommand{\Bias}{\mathrm{Bias}}

\begin{document}

\maketitle

\pagebreak

\begin{homeworkProblem}
\textbf{(a)}
$q \land p \to \lnot r$

\textbf{(b)}
If the car won't start, then neither there is water in the cylinders nor the head gasket is blown.
\end{homeworkProblem}

\begin{homeworkProblem}
\textbf{(a)}
$\lnot r \to \lnot (p \lor q)$

\textbf{(b)}
If you are in Kwangju, then you are in South Korea but not in Seoul.
\end{homeworkProblem}

\begin{homeworkProblem}
\textbf{(i)}
$a\iff \top,b\iff \top$
\begin{align*}
(a \lor b) \land (\lnot (a\land b)) &\iff (\top \lor \top) \land \lnot (\top \land \top) \\
&\iff \top \land \lnot \top \\
&\iff \top \land \bot \\
&\iff \bot
\end{align*}
\begin{align*}
(a\leftrightarrow \lnot b) &\iff (\top \leftrightarrow \lnot \top) \\
&\iff (\top \leftrightarrow \bot) \\
&\iff \bot
\end{align*}

We can calculate all of the answers in this way and make a table \ref{tab:1}.

\begin{table}[ht!]
	\centering
	\begin{tabular}{c | c || c | c}
		$a$ & $b$ & $(a \lor b) \land (\lnot (a\land b))$ & $a\leftrightarrow \lnot b$ \\
		\hline
		T & T & F & F \\
		T & F & T & T \\
		F & T & T & T \\
		F & F & F & F
	\end{tabular}
	\caption{truth table}
	\label{tab:1}
\end{table}
If you look at the third and fourth column of the table \ref{tab:1}, then you can tell that the result of the two expressions are same. Therefore, you can say that $(a \lor b) \land (\lnot (a\land b))$ and $a\leftrightarrow \lnot b$ are logically equivalent. \qed
\end{homeworkProblem}

\begin{homeworkProblem}
A sufficient condition for the given statement: $x$ is a multiple of four.
\\

\solution

If $x=4k$ ($k$ is an integer), then $x/2 = 2k$, which means that $x/2$ is an even integer.
\end{homeworkProblem}

\begin{homeworkProblem}
The result of $\lnot (p \land q)$ can be deduced easily using the already-known value $p\land q$ which is shown in fourth column in the table \ref{tab:2}.
\begin{table}[ht!]
	\centering
	\begin{tabular}{c | c || c | c | c}
		$p$ & $q$ & $p\uparrow q$ & $p\land q$ & $\lnot (p \land q)$ \\
		\hline
		T & T & F & T & F \\
		T & F & T & F & T \\
		F & T & T & F & T \\
		F & F & T & F & T
	\end{tabular}
	\caption{truth table}
	\label{tab:2}
\end{table}

The results of $p\uparrow q$ and $\lnot (p \land q)$ are exactly same. Therefore, we can conclude that $p\uparrow q$ is logically equivalent to $\lnot (p \land q)$. \qed
\end{homeworkProblem}

\begin{homeworkProblem}
\part

\begin{table}[ht!]
	\centering
	\begin{tabular}{c | c || c | c }
		$p$ & $q$ & $(p\uparrow q)\uparrow(p\uparrow q)$ & $p\land q$ \\
		\hline
		T & T & T & T \\
		T & F & F & F \\
		F & T & F & F \\
		F & F & F & F
	\end{tabular}
	\caption{truth table}
	\label{tab:3}
\end{table}

The third and the fourth column on table \ref{tab:3} is identical, which means $(p\uparrow q)\uparrow(p\uparrow q)$ and $p\land q$ are logically equivalent. \qed
\\

\part

\begin{table}[ht!]
	\centering
	\begin{tabular}{c | c || c | c | c | c }
		$p$ & $q$ & $p\uparrow p$ & $q\uparrow q$ & $(p\uparrow p)\uparrow(q\uparrow q)$ & $p\lor q$ \\
		\hline
		T & T & F & F & T & T \\
		T & F & F & T & T & T \\
		F & T & T & F & T & T \\
		F & F & T & T & F & F
	\end{tabular}
	\caption{truth table}
	\label{tab:4}
\end{table}

The sixth and the seventh column on table \ref{tab:4} is identical, which means $(p\uparrow p)\uparrow(q\uparrow q)$ and $p\lor q$ are logically equivalent. \qed
\\

\part

\begin{table}[ht!]
	\centering
	\begin{tabular}{c | c || c | c }
		$p$ & $q$ & $p\uparrow(q\uparrow q)$ & $p\to q$ \\
		\hline
		T & T & T & T \\
		T & F & F & F \\
		F & T & T & T \\
		F & F & T & T
	\end{tabular}
	\caption{truth table}
	\label{tab:5}
\end{table}

The sixth and the seventh column on table \ref{tab:4} is identical, which means $(p\uparrow p)\uparrow(q\uparrow q)$ and $p\lor q$ are logically equivalent. \qed
\end{homeworkProblem}

\begin{homeworkProblem}
\begin{table}[ht!]
	\centering
	\begin{tabular}{l | c}
		Statement & Reasons \\
		\hline
		1. $p\land (q\lor r)$ & given \\
		2. $\lnot(p\land q)$ & given \\
		\hline
		3. $\lnot p\lor \lnot q$ & De Morgan's law \\
		4. $\lnot q\lor \lnot p$ & commutativity \\
		5. $q\to \lnot p$ & implication \\
		6. $p$ & simplification, 1 \\
		7. $\lnot (\lnot p)$ & double negation \\
		8. $\lnot q$ & \textit{modus tollens}, 6, 5 \\
		9. $(q\lor r)\land p$ & commutativity, 1 \\
		10. $q\lor r$ & simplification \\
		11. $r\lor q$ & commutativity \\
		12. $\lnot(\lnot r)\lor q$ & double negation \\
		13. $\lnot r\to q$ & implication \\
		14. $\lnot(\lnot r)$ & \textit{modus tollens}, 8, 13 \\
		15. $r$ & double negation \\
		16. $p\land r$ & conjunction, 6, 15 
	\end{tabular}
	\caption{proof sequence}
	\label{tab:6}
\end{table}
\end{homeworkProblem}

\begin{homeworkProblem}
\begin{table}[ht!]
	\centering
	\begin{tabular}{l | c}
		Statement & Reasons \\
		\hline
		1. $p$ & given \\
		2. $p\to r$ & given \\
		3. $q\to \lnot r$ & given \\
		\hline
		4. $\lnot q \lor \lnot r$ & implication \\
		5. $\lnot r \lor \lnot q$ & commutativity \\
		6. $r\to \lnot q$ & implication \\
		7. $r$ & \textit{modus ponens}, 1, 2 \\
		8. $\lnot q$ & \textit{modus ponens}, 7, 6
	\end{tabular}
	\caption{proof sequence}
	\label{tab:7}
\end{table}
\end{homeworkProblem}

\begin{homeworkProblem}
Is $a\to \lnot a$ a contradiction?
\\

\solution

No, it is not. If we have $\bot$ for $a$, then the given statement is $\bot \to \top$, which is true (not a contradiction). \qed
\end{homeworkProblem}
\end{document}