\documentclass{article}

\usepackage{fancyhdr}
\usepackage{extramarks}
\usepackage{amsmath}
\usepackage{amsthm}
\usepackage{amsfonts}
\usepackage{amssymb}
\usepackage{tikz}
\usepackage[plain]{algorithm}
\usepackage{algpseudocode}

\usetikzlibrary{automata,positioning}

%
% Basic Document Settings
%

\topmargin=-0.45in
\evensidemargin=0in
\oddsidemargin=0in
\textwidth=6.5in
\textheight=9.0in
\headsep=0.25in

\linespread{1.1}

\pagestyle{fancy}
\lhead{\hmwkAuthorName}
\chead{\hmwkClass\ (\hmwkClassInstructor): \hmwkTitle}
\rhead{\firstxmark}
\lfoot{\lastxmark}
\cfoot{\thepage}

\renewcommand\headrulewidth{0.4pt}
\renewcommand\footrulewidth{0.4pt}

\setlength\parindent{0pt}

%
% Create Problem Sections
%

\newcommand{\enterProblemHeader}[1]{
    \nobreak\extramarks{}{Problem \arabic{#1} continued on next page\ldots}\nobreak{}
    \nobreak\extramarks{Problem \arabic{#1} (continued)}{Problem \arabic{#1} continued on next page\ldots}\nobreak{}
}

\newcommand{\exitProblemHeader}[1]{
    \nobreak\extramarks{Problem \arabic{#1} (continued)}{Problem \arabic{#1} continued on next page\ldots}\nobreak{}
    \stepcounter{#1}
    \nobreak\extramarks{Problem \arabic{#1}}{}\nobreak{}
}

\setcounter{secnumdepth}{0}
\newcounter{partCounter}
\newcounter{homeworkProblemCounter}
\setcounter{homeworkProblemCounter}{1}
\nobreak\extramarks{Problem \arabic{homeworkProblemCounter}}{}\nobreak{}

%
% Homework Problem Environment
%
% This environment takes an optional argument. When given, it will adjust the
% problem counter. This is useful for when the problems given for your
% assignment aren't sequential. See the last 3 problems of this template for an
% example.
%
\newenvironment{homeworkProblem}[1][-1]{
    \ifnum#1>0
        \setcounter{homeworkProblemCounter}{#1}
    \fi
    \section{Problem \arabic{homeworkProblemCounter}}
    \setcounter{partCounter}{1}
    \enterProblemHeader{homeworkProblemCounter}
}{
    \exitProblemHeader{homeworkProblemCounter}
}

%
% Homework Details
%   - Title
%   - Due date
%   - Class
%   - Section/Time
%   - Instructor
%   - Author
%

\newcommand{\hmwkTitle}{Homework\ \#1}
\newcommand{\hmwkDueDate}{September 27, 2016}
\newcommand{\hmwkClass}{CS300}
\newcommand{\hmwkClassInstructor}{Prof. Sunghee Choi}
\newcommand{\hmwkAuthorName}{Ohjun Kwon}

%
% Title Page
%

\title{
    \vspace{2in}
    \textmd{\textbf{\hmwkClass:\ \hmwkTitle}}\\
    \normalsize\vspace{0.1in}\small{Due\ on\ \hmwkDueDate\ at 10:30am}\\
    \vspace{0.1in}\large{\textit{\hmwkClassInstructor}}
    \vspace{3in}
}

\author{\textbf{20160051 \hmwkAuthorName}}
\date{}

\renewcommand{\part}[1]{\textbf{\large Part \Alph{partCounter}}\stepcounter{partCounter}\\}

%
% Various Helper Commands
%

% Useful for algorithms
\newcommand{\alg}[1]{\textsc{\bfseries \footnotesize #1}}

% For derivatives
\newcommand{\deriv}[1]{\frac{\mathrm{d}}{\mathrm{d}x} (#1)}

% For partial derivatives
\newcommand{\pderiv}[2]{\frac{\partial}{\partial #1} (#2)}

% Integral dx
\newcommand{\dx}{\mathrm{d}x}

% Alias for the Solution section header
\newcommand{\solution}{\textbf{\large Solution}}

% Probability commands: Expectation, Variance, Covariance, Bias
\newcommand{\E}{\mathrm{E}}
\newcommand{\Var}{\mathrm{Var}}
\newcommand{\Cov}{\mathrm{Cov}}
\newcommand{\Bias}{\mathrm{Bias}}

\begin{document}

\maketitle

\pagebreak
\begin{homeworkProblem}
\part

$T(n)=T(n-1)+\Theta(n)$
\\

\part

We can use unrolling method to solve this recurrence as below.
\begin{align*}
T(n) &= T(n-1) + cn \\
&= T(n-2) + c(n-1) + cn \\
&\ \vdots \\
&= \sum^n_{i=1} ci \\
&= c\sum^n_{i=1} i \\
&= c\frac{n(n-1)}2 \\
&= c_2n^2+c_1n+c_0 \in \Theta(n^2)
\end{align*}
where $c_2$, $c_1$, $c_0$ are constants. \qed
\end{homeworkProblem}

\begin{homeworkProblem}
\part

In order to be $A'$ is called a sorted permutation of $A$, we need to show that $A'$ is made of the initial array $A$, and it is put in order that satisfies $\forall i, j\ \mbox{ s.t. } 1 \leq i \leq j \leq n, A[i] \leq A[j]$ on the termination.
\\

\part

\paragraph{Loop invariant}
At the beginning of each iteration of the loop, the element $A[j]$ is the smallest element in the subarray $A[j...n]$ where $n=A.length$. Also, these values are originated from the initial subarray $A[j...n]$.

\paragraph{Initialization}
The loop starts with $j=n$. The subarray $A[j...n]=A[n...n]$ only contains one element, which is apparently the smallest value in the subarray.

\paragraph{Maintenance}
The smallest element in the subarray $A[j...n]$ is $A[j]$ by the loop invariant stated above. Line 4 in the \textit{BUBBLESORT} is executed only if the element $A[j-1]$ is bigger than $A[j]$. This extends our range $j...n$ that loop invariant condition($A[j-1]$ is the smallest element in the subarray $A[j-1...n]$, and $A[j-1...n]$ is a permutation of the previous sequence) holds to $j-1...n$, which maintains the loop invariant on the next loop.

\paragraph{Termination}
The loop terminates when the variable $j$ reaches $i$, which makes $A[i]$ minimum in the subarray $A[i...n]$ and the subarray $A[i...n]$ is a permutation of the subarray $A[i...n]$ at the initial point.
\\

\part

\paragraph{Loop invariant}
At the beginning of each iteration of the loop, the subarray $A[1...i-1]$ is sorted array that contains $i-1$ smallest elements of the original array $A[1...n]$. In addition, the subarray $A[i...n]$ contains the remaining values waiting to be sorted, i.e. $A[1...n]$ is a permutation of the previous values.

\paragraph{Initialization}
The loop starts with $i=1$. The subarray $A[1...0]$ is an empty array, which is considered as sorted(is a permutation and formed a sorted sequence).

\paragraph{Maintenance}
At the beginning of the loop, it begins with $A[1...i-1]$ sorted, which contains $i-1$ smallest values. After the execution of the 2-4th line in \textit{BUBBLESORT}, the smallest element in the subarray $A[i...n]$ will be located at $A[i]$, and the new subarray $A[i...n]$ is a permutation of the previous $A[i...n]$ as we proved at \textbf{Problem 2 Part B}. As $A[i]$ is the $i$-th smallest element in $A[1...n]$, the $A[1...i]$ is new subarray that satisfies the loop invariant.

\paragraph{Termination}
The loop terminates when the variable $i$ reaches $n$, and the loop invariant says that $A[1...n-1]$ contains $n-1$ smallest elements in increasing order, in other words, sorted. This ensures the last value $A[n]$ is the biggest value in the array $A[1...n]$, which also satisfies the ordering, and all elements are from the original array.
\\

\part

The worst case for the bubblesort is the "backward sorted sequence". It makes the comparison and exchange happens everytime, which is executed $n-1$, $n-2$, $\cdots$, $1$ times. (Total $\frac{n(n-1)}2$ times). Therefore, $\Theta(n^2)$ is the worst-case time complexity, which is same with insertion sort. 
\end{homeworkProblem}

\begin{homeworkProblem}
Prove: For any two functions $f(n)$ and $g(n)$, we have $f(n)=\Theta(g(n))$ if and only if $f(n) = O(g(n))$ and $f(n) = \Omega(g(n))$
\\

\solution

($\Rightarrow$)

$f(n)=\Theta(g(n))$ means that $f(n)$ satisfies the statement~\ref{eq:theta}.
\begin{equation}
\exists c_1,c_2,n_0 > 0\ \mbox{ s.t. } \forall n\geq n_0, 0\leq c_1g(n)\leq f(n)\leq c_2g(n)
\label{eq:theta}
\end{equation}
In addition, the statement~\ref{eq:theta} can be broken into two following statements.
\begin{align}
\exists c_2,n_0 > 0\ \mbox{ s.t. } \forall n\geq n_0, 0\leq f(n)\leq c_2g(n) \label{eq:o} \\
\exists c_1,n_0 > 0\ \mbox{ s.t. } \forall n\geq n_0, 0\leq c_1g(n)\leq f(n)
\label{eq:omega}
\end{align}
The statement~\ref{eq:o} is the definition of the $f(n)=O(g(n))$, and the statement~\ref{eq:omega} is the definition of the $f(n)=\Omega(g(n))$. These conclusions were derived from the given statement (statement~\ref{eq:theta}). \qed
\\

($\Leftarrow$)

This way also can be proved by doing the proof exactly opposite way.
$f(n)=O(g(n))$ gives us the statement~\ref{eq:o}, and $f(n)=\Omega(g(n))$ gives us the statement~\ref{eq:omega}.

Next, if we combine those two statements into one, it looks like the statement~\ref{eq:theta}, which means $f(n)=\Theta(g(n))$. \qed
\end{homeworkProblem}
\begin{homeworkProblem}
\part

$T(n)=2T(n/4)+\sqrt{n}$

\solution

$a=2$, $b=4$ $\Rightarrow$ $n^{\log_ba}=\sqrt n$; $f(n)=\sqrt n$

Case 2 on the master method: $f(n)=\Theta(\sqrt n \mbox{lg}^0 n)$, $k=0$

$\therefore T(n)=\Theta(\sqrt n \mbox{lg} n)$
\\

\part

$T(n)=T(n-2)+n^2$

\solution
\begin{align*}
T(n)&=n^2+T(n-2) \\
&=n^2+(n-2)^2+T(n-4) \\
&=\sum_{i=0}^{n/2}(n-2i)^2 \\
&=\sum_{i=0}^{n/2}(n^2-4ni+4i^2) \\
&=\frac{n^3}2-4n\frac{\frac n2\left( \frac n2-1\right)}2+4\frac{\frac n2\left(\frac n2+1\right)\left(2\frac n2+1\right)}6 \\
&\in \Theta(n^3)
\end{align*}
$\therefore T(n)=\Theta(n^3)$
\\

\part

$T(n)=7T(n/3)+n^2$

\solution

$a=7$, $b=3$ $\Rightarrow$ $n^{\log_ba}=n^{\log_37}$; $f(n)=n^2$

Case 3 on the master method: $f(n)=\Omega(n^{log_37+\epsilon})$, for some $\epsilon>0$ ($\because log_37<2$)

$7(n/3)^2\leq cn^2$ for $c=7/9$

$\therefore T(n)=\Theta(n^2)$
\\

\part

$T(n)=2T(n/2)+3T(n/3)+n^2$

\solution


\end{homeworkProblem}
\end{document}